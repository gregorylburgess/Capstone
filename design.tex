\chapter{Design}
\label{design}
\section{Program Requirements}
\label{programRequirements}

\subsection{Motivation}
While detriments to SAON technologies are well documented \cite{Akbarzadeh2013} \cite{Heupel2006} \cite{Howard2002}  \cite{Kessel2015} \cite{Steel2014}, there few tools/services to analytically design SAONs around them.  Furthermore, none of these tools/services are free and open-source.


\subsubsection{Cost Efficiency}
\label{motivationCost}
Section~\ref{CostAltTech} discusses the costs of marine telemetry systems, noting that acoustic telemetry systems produce data at a significantly lower ($\ge$10x cheaper) cost than VHF or GPS/Satellite based technologies.  In order to maintain the cost-efficiency of acoustic technology, a significant number of detections $\ge$~10$\%$ of the produced transmissions must be captured by the SAON's receiver array.  Given the numerous (but avoidable) impediments to reception of these acoustic signals (\ref{RulesOfThumb}), the array-design process becomes critical to maintaining the cost-efficiency of SAON technologies.  A free network design tool would help to maintain the cost-efficiency of SAONs by eliminating costs surrounding their design and evaluation, and increasing their data recovery rates.  


\subsubsection{Metrics}
\label{motivationMetrics}
The computation of network metrics (Absolutes Recovery Rate, Unique Recovery Rate, Network Sparsity) is very labor intensive at large scale.  Additionally, the process of computation may vary from experiment to experiment.  An automated tool would solve both issues by providing a fast, repeatable, reliable, and well-documented method for computation.  Metrics from such a tool would be useful in directly comparing different network designs.


\subsubsection{Transparency}
\label{motivationTransparency}
An open-sourced tool/service would make the design process more transparent, permitting peer-review and modification.  This would provide increased confidence in the process, and increased adoption of the tool.  Increased adoption would result in a larger number of efficient SAONs, leading to higher data recovery rates, better data quality, increased return-on-investment, and the ability to better address research questions.


\subsection{Supported Workflows}
\label{workflows}
\subsubsection{Static Analysis}
\label{staticAnalysis}
As mentioned in section~\ref{motivationMetrics}, a primary motive for this tool was the ability to create a repeatable means of measuring the performance of a SAON.  To this end, the ability to measure an existing network design is important.  Users should be presented with network metrics after specifying bathymetry, receiver locations, network properties, and an animal model for a given study site.


\subsubsection{Optimal Design}
\label{optimalDesign}
The primary motive for this tool is the ability to design optimal SAONs.  Users should be presented with a network design (optimal receiver locations), and network metrics after specifying bathymetry, the number of receivers in the network, network properties, and an animal model for a given study site.


\subsubsection{Optimal Addition}
\label{optimalAddition}
Similar to the problem of optimal design, is the problem of optimal addition: the augmentation of an already existing SAON.  Users should be presented with a network design (optimally augmenting receiver locations), and network metrics after specifying bathymetry, the number of receivers to add to the network, network properties, existing receiver locations, and an animal model for a given study site.

\section{Conceptual Model}
\label{conceptualModel}
\subsection{Time/Space Modeling}
\label{timeSpaceModel}
\subsubsection{Spatial Modeling}
\label{spatialModeling}
To model various attributes of a 3-dimensional underwater environment, several two-dimensional Grid of cells (in the x and y dimensions) containing numerical values are used.  Numerical values in those cells, combined with user-defined values, can then be used in various shape functions to generate a third dimension (z) of values for that cell.  In this way, a significant amount of memory is saved by computing values for a specific three-dimensional cell as needed, instead of storing an additional dimension of values.  

%%TODO FIX
\subsubsection{Temporal Modeling}
\label{temporalModeling}
Modeling a 3-dimensional environment over time is computationally intensive, as individual values can vary with time.  The primary temporally-dependent phenomena considered by this system is that of animal movement.  The animal model does not represent individual animal movement at a specific time, but the summary cell-residency of all tagged individuals over the entire study period.  As a result, temporally-dependent phenomena need not be considered. 




\subsection{Bathymetric Modeling}
\label{bathymetyricModeling}
\subsubsection{Bathymetric Grid}
\label{bathymetricGrid}
Bathymetry files are generally given as two-dimensional matrix of numerical values or a list of x,y,z values.  Bathymetric files describe a three-dimensional space as a regular grid of rectangular prisms (cells) with constant length (x-dimension) and width (y-dimension), but varying negative (depth is negative)  heights (z-dimension).  The resolution of a bathymetric file is given by the x and y (length and width) dimensions of its cells.  For example, a 50 meter Bathymetry file has cell sizes of approximately 50 meters square (although these cells are not necessarily perfectly square).  Bathymetric files include meta-data listing the file's beginning and ending coordinates (North/South Latitudes and East/West Longitudes), the grid size (the number of rows and columns) in cells, and grid resolution (either in linear distance or degrees of latitude/longitude).

MANDe works on a two-dimensional, grid-based system, taking advantage of the grid given by the user-provided bathymetric file.  The Bathymetric Grid is a 2D Grid containing numerical values that describe a third dimension (depth).  The exact spatial extent and resolution of this description are given by the bathymetric file.  Thus, the resolution of MANDE's simulation and output are dictated by the resolution of the input bathymetric file. 

A cell $x$ (with row index $i$ and column index $j$) on the Bathymetry Grid ($A$) is referred to as $A_x$ or $A_{(i,j)}$.

\subsubsection{Bathymetric Filetypes}
\label{bathymetricFiletypes}
Two highly popular file formats for Geographical Information Systems are provided by NetCDF and ArcGIS.  NetCDF provides an open source file format that lists a header of meta-data and a white-space delimited matrix of numerical values.  ArcGIS is a private institution that supplies many different file types, formats, and encodings for a family of GIS-related software systems.  ArcGIS also supports the encoding and transcription of its proprietary formats to the NetCDF format.  Due to the large number of possible format and encoding combinations in ArcGIS file formats and the ability to translate file these various formats to NetCDF, MANDe supports the NetCDF standard by default.


\subsubsection{Bathymetric Resolution}
Two key components of this system are the animal model and the bathymetric shadowing model.  These models make decisions based upon the depth at a particular cell and the distance between cells, data which is governed by the input bathymetry file.  As stated in Section~\ref{bathymetricGrid}, the resolution of the program's output is dependent upon the resolution of the input bathymetry file.  Obviously, higher resolution grids will offer higher resolution results; but, high-resolution bathymetric files tend to be difficult to come by.  These files are often held by private agencies or simply never released to the public.  It may then seem useful to artificially increase the resolution of the simulation by dividing the input file's bathymetric cells into sub-cells of finer resolution, but doing so increases the computational size of the program without meaningfully increasing the accuracy of the results.    
 
 
When subdividing cells, either the sub-cells are given the same depth as their parent cell, or the depth of a sub-cell is interpolated from surrounding cells by some smoothing function.  Subdividing a cell into sub-cells with the same depth makes the assumption that all sub-cells are actually the same depth.  Furthermore, this results in the animal and bathymetric shadowing models making the same depth-based decision for all sub-cells that they would for the larger parent cell, needlessly increasing the computational load (see Figure~\ref{duplicate}).  Subdividing a cell into sub-cells with a depth governed by a smoothing function makes the assumption that there are no impeding obstacles between neighboring cells, and that there is a smooth transition between them (see figure~\ref{smooth}).  Subdividing the two cells in Figure~\ref{LoS} half (ignoring for now the y component of the grid) results in the four sub-cells with depths given by a smoothing function (Figure~\ref{smooth}) would result in the large depth change at the sheer cliff face in Figure~\ref{LoS} being smoothed into smaller changes in depth, which would allow the unobstructed transmission of acoustic signals.


Both strategies (duplicating depth and applying a smoothing function) for artificially increasing the resolution of a bathymetric file disregard the manner in which the bathymetry was originally observed.  Bathymetry is almost always computed as the average observed depth at several points within a geographic area of a given size (resolution).  For example, imagine a particular cell in a bathymetric grid has a steep cliff running across the middle.  Assuming that the sea floor at the top and base of the cliff were perfectly flat, and one measurement was taken at the top of the cliff and one at the bottom, the cell would have a depth equal to the average of the two observed depths.  This average depth would then represent the depth for that entire cell, modeling it as a perfectly flat surface.  Obviously this is problematic as the true nature of the sea floor is misrepresented.  This misrepresentation leads to two conflicting arguments, the first is that the application of smoothing functions or duplicating depths of already averaged data makes faulty assumptions about real-world bathymetry.  On the other hand, because source bathymetric files already represent aggregate data, one could argue that any conclusions drawn from the source bathymetry files are already faulty.  Therefore, the conclusions one can draw are only as good as the bathymetric information available.  As the resolution of measured bathymetry (bathymetric measurements taken from the real world) increases (becomes finer), so too does the accuracy of the simulation.  While artificially increasing the resolution of a source bathymetric file may skew results, it is sometimes useful for meshing two bathymetric source files of varying resolution into one larger bathymetric dataset.  This program does not currently support modifying the resolution of source bathymetric files.

\begin{figure}[ht]
	\begin{subfigure}{.5\textwidth}
		\centering
		\includegraphics[width=.7\linewidth]{LoS.png}
		\caption{
			\label{resolutionScale}}
		\label{LoS}
	\end{subfigure}%
	\begin{subfigure}{.5\textwidth}
		\centering
		\includegraphics[width=.7\linewidth]{duplicate.png}
		\caption{
			\label{duplicate}}
	\end{subfigure}
	\begin{subfigure}{.5\textwidth}
		\centering
		\includegraphics[width=.7\linewidth]{smooth.png}
		\caption{
			\label{smooth}}
	\end{subfigure}
	\caption{(a) illustrates the bathymetric shadowing model for two adjacent cells within a bathymetric grid.  (b) shows how artifically increasing the resolution of the bathymetric grid from (a) using the duplication method of cell sub-division does not affect the bathymetric shadowing model but increases the number of cells to consider.  (c) shows how artificially increasing the resolution using a smoothing function increased the number of cells to consider and can lead to inflated signal reception.}
\end{figure}


\subsubsection{Bathymetric Shadowing}
\label{bathymetricShadowing}
In the real world, the transmission of an acoustic ping originates from the tagged animal and propagates to the receiver.  This propagation is governed by complex interactions with the surrounding environment (including the bottom substrate, water density, distance to the surface/sea-floor, thermoclines, the number of tagged animals in the vicinity, and ambient noise).  Because it is difficult to model these phenomena without significant data on a large number of variables over a large area, the system uses a simplified propagation model relying on direct line of sight between a tagged animal and a receiver.  Simply put, in order to observe a transmission, there must exist a physically-unobstructed path between a tagged animal and a receiver.  




\subsection{Animal Modeling}
\label{animalModeling}

\subsubsection{Behavior Grid}
\label{behaviorGrid}
Animals exhibit many different movement patterns and habitat preferences (both of which can vary in three-dimensional space).  This greatly affects their distribution within the study space, and thus the network configuration that should be deployed to capture their movement.  To describe the distribution of transmissions released from tagged animals, a two dimensional Grid with the same dimensions and resolution as the Bathymetry Grid (Section~\ref{bathymetricGrid}) is used.  Each cell in this ''Behavior Grid'' ($B$) contains a positive real value indicating, out of all transmissions that will be released over the entire study period, the percentage of transmissions that are expected to be released from within that cell's water column.  This value can loosely be though of as the ''animal-residency'' of a cell.  A cell $x$ (with row index $i$ and column index $j$) on the Behavior Grid ($B$) is referred to as $B_x$ or $B_{(i,j)}$.

To populate this 2D grid with values, two behavioral models are provided to simulate the horizontal animal distribution.  Two more models are provided to simulate the vertical distribution.  $Note$: the terms ''animal'' and ''transmission'' are used interchangeably since acoustic transmissions are the metric being counted, but are emitted from the tags carried by animals.  


\subsubsection{Horizontal Movement Modeling}
\label{animalMovementModel}
To simulate tagged animal movement across the horizontal x/y space (as one would expect to see on a map from above), two basic probabilistic movement models are provided: Random Walk, and Ornstein-Uhlenbeck(OU).  

\paragraph{Random Walk Model}
\label{randomWalkModel}
The Random Walk model assumes that animals move randomly through the environment.  As a result, over the entire study period, each valid grid cell (as defined by the Restricted Vertical Habitat Range (Section~\ref{restrictedVerticalHabitat})) will see an equal amount of animal traffic.  The result is that every valid cell  in the grid will have the same chance of capturing an animal's acoustic transmission.  It is assumed that tagged individuals will be willing and able to very briefly (in probabilistically negligible time frames) pass through inhospitable (over dry land, through impassible terrain) cells to get to other cells.  This means that disjoint sections of habitat will contain an equal share of acoustic transmissions.

\paragraph{Ornstein-Uhlenbeck Model}
\label{ouModel}
The Ornstein-Uhlenbeck(OU) model\cite{OU} supports the idea that over time, animals will tend to congregate near certain points of interest.  This concept models an animal's desire to seek out and remain near a physically significant structure, a region of high food availability, breeding grounds, shelter, etc.  The OU algorithm allows for the modeling of the attraction towards a focal point in the x and y directions separately.  Assuming a center point at the origin of a Cartesian grid, increasing the attraction in the x-direction will bring the distribution closer to the x-axis, and decreasing it will spread the distribution away from the x-axis.  The same holds true for the attraction in the y-direction/y-axis.  A correlation value ($-1\le x \le 1$) allows for tilting the angle of the distribution.  A positive correlation value tilts the distribution clockwise, and a negative correlation value tilts it counter clockwise.  Correlations of 0 (no tilt), 1 ($180^{\circ}$ tilt), and -1 (-$180^{\circ}$ tilt) will have no observable effect on the distribution.


\begin{figure}[ht]
	\label{OUimg}
	\centering
	\includegraphics[scale=0.5]{OUpos7.png}
	\caption{An example of a distribution given by the Ornstein-Uhlenbeck Model with a high attraction value in the x-direction, a low attraction value in the y-direction, and a correlation value of 0.7.}
\end{figure}


\subsubsection{Vertical Movement Modeling}
\label{verticalMovement}
To model the z-axis movement behaviors of animals within the water column (as one would expect to see from the side of an aquarium), two optional movement models are provided: Habitat Preference, and Restricted Vertical Habitat Range.  The first controls the distribution of animals within the water column, and the second prohibits animal residency within cells that fall outside specified depth ranges. 

\paragraph{Habitat Preference}
\label{habitatPref}
Some animals exhibit the preference to reside within a specific section of the water column.  For example, prey animals may prefer hiding in reef heads at the bottom of the water column, while predators will prefer to hover several meters off the bottom.  This preference can be incorporated into the animal model by specifying mean ("Preferred Depth") and standard deviation("SD of Preferred Depth") values.  These values are given as a measure of the distance (in meters) from the bottom.  For example, specifying a depth of '0' for Preferred Depth indicates that the animal prefers to live on the sea floor, while a value of '5' indicates that the animal prefers to live 5m off the sea floor.  Allowing a standard deviation value allows for the modeling of animals that tend to be sedentary within the water column (a small deviation), and those that migrate through the water column (a large deviation).  The simulation does not support the modeling of sub-surface animals, as any part of the distribution curve falling below the sea floor will be redistributed along the rest of the curve. 


\paragraph{Restricted Vertical Habitat Range}
\label{restrictedVerticalHabitat}
Some animals will live only in a specific depth range.  For example, a deep sea fish may live only in depths of 300-400 meters.  To incorporate this into the behavioral model, users can specify a minimum and maximum vertical habitat range for their animal.  If this option is selected, the program will only simulate animals in cells whose maximum depths are between the given minimum and maximum depths.  As in the Random Walk Model, it is assumed that animals are willing and able to move between disjoint areas of habitat.


\subsection{Receiver Modeling}
\label{receiverModel}

\subsubsection{Acoustic Attenuation}
\label{acousticAttenuation}
In practice, the highest probability of observing an acoustic transmission occurs when a tag is several dozen meters away from the receiver (due to CPDI effects \cite{Kessel2015}), and dropping off as the tag moves further away from the receiver.  The attenuation of acoustic signal is modeled as a deteriorating function following a Gaussian distribution.  Given a particular distance between a tag and receiver, this distribution describes the probability of capturing the transmission.  This distribution is defined by user-specified values for mean value, peak value (at the mean),  and standard deviation.  Within this framework, CPDI effects can be modeled by modifying the mean value.  Herein, the probability of detecting an unobstructed acoustic transmission from $x$ meters away is denoted as $P_{Attenuation}(x)$.

\subsubsection{Detection Range}
\label{detectionRange}
The distance at which a transmission from an acoustic tag can be captured is largely determined by the transmission power of the acoustic tag.  It is assumed that all animals have the same model of acoustic tag and that "transmission range" is instead a property of the receiver, referred to as "Detection Range" ($D_{range}$).  The Detection Range of a receiver is defined to be the average distance (specific to the given study site) at which the probability of detecting an acoustic transmission drops to 5\%.

\subsubsection{Detection Area}
\label{detectionArea}
The Detection Area of a receiver is defined as the square plane of cells around a receiver which will be considered when evaluating a potential receiver emplacement.  Theoretically, the Detection Area of a receiver is a circle with a radius equal to the receiver's Detection Range ($D_{range}$).  The Detection Area is defined as a square Grid of cells with an edge length ($d_{detection}$) equal to 2*$D_{range}$ + 1, centered on the cell where a receiver will theoretically be placed.  The additional cell ensures odd dimensions for the square, so that there exists a central cell to model as the receiver's location. A square is used instead of a circle for simplicity of book keeping (the program simply keeps track of the $d_{detection}$, and the coordinates of the Grid's top left cell).  The theoretical Detection Area can then be imagined as a circle circumscribed in the square Detection Area.  It might seem that this simplification would alter the probabilistic distribution models as additional "gray" cells (those cells outside the circle but within the square) are being considered.  However, this is not the case as those models utilize normal functions based on the absolute distance from the receiver.  Gray cells will therefore contribute exponentially less to the total probability than those cells within the Detection Range, and their effect on the final probability will be negligible.  

\subsubsection{Network Specification}
There are three distinct ways to place receivers into the model: user specification, optimal placement, and optimal projection.  User Specified Receivers (USR) represent receivers that are already deployed at the study site and are being integrated into a larger network.    Program Placed Receivers (PPRs) are receivers that will be  optimally placed by the program, taking into account existing (user specified) receivers placements using the suppression dynamic explained in section~\ref{suppression}.  Projected Receivers are PPRs that do not count towards the network statistic rates returned by the program.  Instead, their contributed recovery rates are given separately, and represent the marginal benefits of placing incrementally more receivers.  


\section{Evaluation of Receiver Emplacements}
\label{evaluationOfReceierEmplacements}
Section~\ref{animalModeling} discusses the models used to simulate animal movement.  These animal models populate the "Behavior Grid" ($B$), a 2D grid (of the same size as the specified Bathymetry Grid) which gives a distribution summary of transmissions across the study area, and a shape function which describes the distribution of animals within the water column at various depths.  Section~\ref{LoS} discusses the Line of Sight (LoS) model used to simulate the obstruction of acoustic transmission.  Section\ref{receiverModel} discusses how attenuation affects the propagation of acoustic signals and thus the probability of detecting acoustic transmissions from a particular distance ($P_{Attenuation}(x)$).  Together, these models are used to evaluate receiver locations throughout the study site.  

\subsection{Goodness Grid}
\label{GoodnessGrid}
Recall that MANDe models a 3D environment (the study site).  To avoid evaluating every cell in this 3D environment as a receiver placement, MANDe assumes that all receivers in the study will have the same elevation off the sea floor.  This assumption of a fixed receiver elevation (of $k$ meters) reduces the search space to a 2D surface parallel to the sea floor.  Thus, it makes sense to store the chosen metric for evaluation of this search space in a 2D Grid with the same dimensions and resolution as the Bathymetry Grid.  This Gird of metrics is referred to as the "Goodness Grid" ($G$).  A cell $x$ (with row index $i$ and column index $j$) on the Goodness Grid ($G$) is referred to as $G_x$ or $G_{(i,j)}$. 


\subsection{Evaluation Algorithms (Bias)}
\label{evalAlgorithms}
The Evaluation or “Goodness” algorithms define a process for evaluating the goodness of a particular cell as a receiver location.  Given a particular cell at row $i$ and column $j$, an Evaluation Algorithm will compute a non-negative, rational value representing the quantity of observed transmissions that a receiver placed $k$ meters off the bottom of cell $(i,j)$ would recover.

While users are able to write their own Evaluation Algorithms, three basic algorithms are provided.  All Evaluation Algorithms compute the Goodness of a potential receiver location ($G_{i,j}$) by summing the number of Estimated Receivable Transmissions (ERT) from all cells within Detection Range of cell ($i,j$).  Each Evaluation Algorithm has a particular bias in its approach to computing ERT.  $ERT(x,T,a,b)$ denotes the ERT value for the cell (a,b), relative to the receiver in cell $T$, as determined by Evaluation Algorithm $x$.  Note: within this paper, the terms "bias" and "algorithm" are used interchangeably when referring to Evaluation Algorithms.

Explicitly, $G_{i,j} = \Sigma_{a=i_0}^{i_n} \Sigma_{b=j_0}^{j_n} ERT(x,(i,j),a,b)$ , where $x$ is the chosen Evaluation Algorithm/Bias, $i_0$ and $j_0$ represent the top-left row/column indexes (respectively) for the Detection Area, and $i_n$ and $j_n$ represent bottom-right row/column indexes (respectively) for the Detection Area.  
\newline\textbf{Note:} here the receiver-containing cell $T$ is denoted as ($i,j$).

\subsubsection{Animal Only (Option “1”)}
\label{bias1}
This option prefers to place receivers in areas of high animal activity, completely oblivious to the surrounding bathymetry.  This is useful for when no bathymetric information is available or transmissions are unlikely to be blocked by bathymetry (in a very flat area, or when animal activity occurs well above the sea floor).  The "Animal Only" option computes ERT for a cell as the animal-residency of a cell ($a,b$)(according to the Behavior Grid), multiplied by the probability of detection due to attenuation for that cell's distance from the receiver cell($T$).\newline
Explicitly,
$ERT(1,T,a,b) = B_{a,b} * P_{Attenuation}(Range(T,a,b))$

\subsubsection{Visible Fish (Option “3”)}
\label{bias3}
This option chooses receiver locations that have the best view of areas with high animal residency.  Both animal presence and visibility due to topography are considered.  Figure~\ref{observableAnimals} depicts this idea.  The green and red normal curves represent the distribution of animals in the water-column.  The green portion of the normal curve illustrates the portion of observable (due to bathymetry) animals.  

This option is most useful when both well-documented animal behavior and high resolution bathymetry are available.  ERT is computed as animal-residency (according to the Behavior Grid), multiplied by the percentage of observable transmissions (due to bathymetric obstruction), multiplied by the probability of detection due to attenuation.  \newline
Explicitly,
$ERT(3,T,a,b) =  B_{a,b} * P_{observation}(T,a,b) * P_{Attenuation}(Range(T,a,b))$

\begin{figure}[ht]
	\centering
	\includegraphics[scale=0.3]{ObservableAnimals.png}
	\caption{An illustration of how ray tracing and integration over a shape function can be used in Option 2 \& Option 3 to compute the probability of detecting fish within a cell.  Dotted lines indicate the maximum and minimum depths visible to the receiver.  The normal distributions in green/red indicate the distribution of fish within a given cell as determined by a shape function.  The green portion of the curve indicates the observable section of the distribution, while the red indicates the unobservable section.  The observable section (green) is computed by integration over the shape function.
		\label{observableAnimals}}
\end{figure}

\subsubsection{Topography Only (Option “2”)}
\label{bias2}
This option places receivers in areas that have the best visibility of the surrounding area, regardless of the expected animal-residency.  This is useful for experiments where animal habitat is unknown or to be determined.  Here, ERT is computed as the number of observable transmissions from a uniformly distributed animal model.  While this algorithm is referred to as "Topography Only", it is implemented using the "Visible Fish/Option 3" algorithm with a simplified animal model.  Specifically, it is assumed that animals are uniformly distributed throughout the environment.  Because animals are uniformly distributed, the more of the water column a receiver can see, the more transmissions it will receive.  Thus, the algorithm will value cells with the best possible view of the surrounding water columns.\newline
Explicitly,
$ERT_{2,T,i,j} =  B_{a,b} * P_{observation}(T,a,b) * P_{Attenuation}(Range(T,a,b))$\newline
Note: With this option, B is initially described as a uniform distribution.


\subsection {Line of Sight Evaluation}
\label{LoSAlgorithm}
Section~\ref{bathymetricShadowing} explains the concept of Bathymetric Shadowing, which requires a bathymetrically unobstructed Line of Sight (LoS) to exist between an acoustic tag and a receiver in order for transmissions from that tag to be received.  Section~\ref{bias3} discusses how the evaluation algorithms utilize this concept to compute the quantity of transmissions that are receivable (ERT).  For this computation, the evaluation algorithm requires knowledge of the deepest depth ($D_{max}(p,q)$) in a cell $q$ that is visible to a receiver in cell $p$.  This depth is used by Evaluation Algorithms 2 and 3 to compute the finite integral for their ERT values.

To determine the $D_{max}$ for a cell $q$, relative to a receiver in cell $p$, the cells could potentially obscure LoS between $p$ and $q$ must first be identified. A list of $n$ such potentially obscuring cells is referred to as $C$.   Figure~\ref{LoS} illustrates this process.  The shaded cells in the left image potentially obstruct the vision from cell $p$ to cell $q$.  Recall that $B_x$ refers to the depth of the cell $x$ as given by the Bathymetry Grid.

Next, the slope between the receiver at $p$ and each cell in $C$ is determined.  The slope between a receiver elevated $k$ meters off the sea floor of cell $p$ and $v$ ($m_{p,v}$) is defined as the difference in elevation between $B_p + k$ and $B_v$ divided by the Euclidean distance ($Edist_{p,v}$) between cell $p$ and cell $v$.  Explicitly, $m_{p,v} = \frac{(B_v - B_p + k)}{Edist_{p,v}}$.  The greatest (largest signed value) slope $m_{p,v}$ for all cells $v$ in $C$ is the Critical Slope ($m_{crit}$).  If a line with this slope were projected from the receiver at $p$ to the target cell $q$, the line would be tangent to the tallest obstruction along that path.  Thus, the critical slope determines $D_{max}$ for cell $q$.  Explicitly, $D_{max} = m_{crit} * Edist_{p,v} + B_p + k$.  Figure~\ref{LoSimage} gives a visual depiction of how $m_{crit}$ is used to determine $D_{max}$.

\begin{figure}[ht]
	\centering
	\includegraphics[scale=0.5]{criticalSlope.png}
	\caption{An illustration of how the critical slope, and $D_{max}$ are computed between cells $p$ and $q$.  The slope between the receiver and all interveening cells are computed (blue and red arrows).  Then, the greatest slope is selected as $m_{crit}$ (dark red arrow) because it poses the greates obstruction to vision of cell $q$.  Finally, $m_{crit}$ is projected into cell $q$ (the light red arrow) to determine $D_{max}$.
		\label{LoSimage}}
\end{figure}


\subsection{Selection of Optimal Emplacements}
\label{selectionOfOptimalEmplacements}
Section~\ref{evaluationOfReceierEmplacements} describes the evaluation of cells as potential receiver locations.  The Optimal Design and Optimal Addition work flows (section~\ref{workflows}) require the identification and selection of a user-specified number of optimal receiver emplacements.  Once all cells within the area of interest have been evaluated, the program selects the user-specified number of optimal receiver locations, and then the user-specified number of projected receivers.


\section{Suppression}
\label{suppression}
As previously mentioned, the optimality of a network depends greatly upon the way in which Data Quality is defined.  Some users will want to design a network that covers as much of a study area as possible, while others might want to heavily saturate a small area with receivers to facilitate higher resolution telemetry.  Still others may wish to find the receiver locations that return the highest number of unique data points.  Unique Data Recovery Rate (UDRR) and Absolute Data Recovery Rate (ADRR) [Section~\ref{dataRecoveryRate}] also play a role in the definition of Data Quality.  To allow for the control of sensor distribution, the Suppression mechanic is provided.


\begin{figure}[ht]
	\centering
	\includegraphics[scale=0.45]{suppression.png}
	\caption{An illustration of the Exact Suppression Algorithm.  (a) depicts a bathymetry grid with infinitely high walls (the white 'h' shape) on an otherwise flat plane (blue region).  (b) depicts a Behavior Grid with a distribution (given by the OU movement model) of animals around the walls.  (c) shows the computed Goodness of Sensor (receiver) locations within the study site.  The program first identifies location 1 (the blue circle) as having the highest unique data recovery rate, and places a receiver there.  The dotted lines represent the receiver's Detection Range.  In (d), the program suppresses the Behavior Grid by subtracting the ERT of each cell within Suppression Range.  Here the Suppression Range Factor is 1.0, so the Suppression Area is the same as the Detection Area.  In (e), the Goodness Grid is recalculated, taking into account the suppressed Behavior Grid.  Additionally, the program identifies location 2 as having the highest Unique Data Recovery Rate.  In (f), the Behavior Grid is again suppressed to account for the placement of receiver (2).
		\label{suppressionImage}}
\end{figure}

\subsection{Suppression Area}
As an input to the program, users can specify a Suppression Range Factor ($f_{supp}$) as a positive real number.  This factor is used to scale the Detection Range ($D_{range}$) (Section~\ref{detectionRange}) and define a Grid similar to the Detection Area (Section~\ref{detectionArea}).  The resulting Grid is referred to as the Suppression Area.  The Suppression Area is centered over a receiver placement and cells within this area have their number of undetected transmissions reduced according to a Suppression Algorithm.  The Suppression mechanic is used during the selection of receiver locations.  After selecting each optimal receiver location (Section~\ref{selectionOfOptimalEmplacements}), the Suppression Area around that receiver emplacement is suppressed.  As a result, receiver emplacement selection is a sequential process, where selecting an emplacement depends upon the suppression of the previous selection.  

Assuming a uniform Goodness Grid, setting the Suppression Range Factor to 1.0 ensures  that receiver Detection Areas do not overlap, maximizing network coverage.  Setting the Suppression Range Factor to 0.5 means that receiver Detection Areas will exactly overlap (the edge of one Detection Area will touch a receiver).  A Suppression Range Factor of 0.0 will cause receivers to be stacked on top of each other, over the highest rated cell in the Goodness Grid.  Using this approach, it is possible to disperse sensors across physical space, while optimizing the number of captured transmissions.

\subsection{Suppression Algorithms}
\label{suppressionAlgorithms}
To promote flexibility, three algorithms for suppression are provided.  The first offers a lower computation time at the cost of divergence from the theoretical model.  The second offers a higher fidelity model, but requires significantly more computation.  The final algorithm provides very high fidelity, but at very high cost.

\subsubsection{Static Suppression}
\label{staticSuppression}
The Static Suppression algorithm takes a "black-out" approach and replaces the goodness of all cells within the Suppression Area by a user-specified value.


\subsubsection{Scalar Suppression}
\label{scaledSuppression}
The Scaled Suppression algorithm reduces the number of undetected transmissions in the Suppression Area using a template.  This algorithm requires a minimum and maximum suppression value ($Supp_{min}$ and $Supp_{max}$).  The algorithm first creates the Suppression Template, a Grid with the same dimensions the Suppression Area.  This Suppression Template is populated with a radial gradient of suppression values.  The central cell in the Suppression Template receives the largest suppression value ($Supp_{max}$), and the cell farthest away from the center receive the smallest suppression value ($Supp_{min}$). Cells between the central cell and the furthest cell receive a suppression value negatively correlated with their distance from the central cell (according to the Suppression Function).  To suppress a particular cell, the  algorithm simply preforms a scalar multiplication (corresponding cells in the Suppression Template and Suppression Area are multiplied together).  Because this computation is very simple, it runs very quickly and is therefore useful for simulations that intend to place a very large number of receivers.  The cost of this computational simplicity is that the algorithm ignores line of sight model.  For instance, a cell whose line of sight to the receiver is obstructed will be reduced by the same factor as a cell near the receiver with an unobstructed line of sight.  Obviously this is not a faithful representation of the conceptual models discussed earlier, but supplied as a computationally-simple alternative.

\subsubsection{Exact Suppression}
\label{exactSuppression}
The primary purpose of this algorithm was to discount transmissions which will be observed, according to an Evaluation Algorithm, by a placed receiver.  This algorithm uses the ERT value given by the Evaluation Algorithms (the user-specified Evaluation Algorithm is used in both Goodness and Suppression calculations) to determine the number of transmissions within a particular cell to discount.  $ERT_{T,i,j}$ denotes the ERT value in cell (i,j) observed from cell $T$.  Exact Suppression works first by computing $ERT_{T,i,j}$ for all cells (i,j)  within the Suppression Area.  Then, each corresponding cell in the Behavior Grid, $B_{i,j}$, is reduced by $ERT_{T,i,j}$.  Finally, the Goodness of all cells within a distance of (Detection Range plus Suppression Range) cells of a suppressed cell (all those that would have their Goodness affected by the suppression) is recalculated.  Figure~\ref{suppressionImage} illustrates the Exact suppression algorithm.  

\section{Optimal Sensor Projection}
In normal research situations, users will have a set number of receivers to place within their study site.  The process of arriving at this number is likely unscientific, perhaps relying on user estimation.  Rather than guessing at the number of receivers to use, and hoping for an adequate data recovery rate, users should be able to calculate the marginal benefit (additional detections, increased Data Recovery Rates) of utilizing a variable number of receivers.   To this end, the program returns the marginal gain in Data Recovery for a given number of additional receivers.  Within the program, users to specify a number of receivers to project, and receive graphs and metrics of the marginal increase in Data Recovery Rate.  With this data, users can determine an appropriate number of receivers to purchase or construct an argument for purchasing more receivers.


\section{Time and Space Complexity}
\label{computationalComplexity}
To evaluate the temporal and spatial complexities of various elements of the program, the following variable inputs to the program are defined:\newline

\noindent\begin{tabularx}{\linewidth}{@{}>{\hsize=.2\hsize}X>{\hsize=1.5\hsize}X@{}}
	$n$ & The square root of the number of cells in the Bathymetry Grid, (the edge length of a square grid).  Also recall that the Bathymetry, Behavior, Goodness, and Coverage Grids are all of identical dimension.\newline\\
	 
	 $D_{range}$ & The Detection Range of a receiver in the simulation..\newline\\
\end{tabularx}



\subsection{Bathymetry Grid}
The bathymetry data for the study site is represented as a Grid of $n^2$ cells.  This data must be copied into local memory (RAM) from the input Bathymetry File, each cell in the grid takes O(1) time to copy, and O(1) space to hold.  Thus, the creation of the Bathymetry Grid will take O(1) * $n^2$ = O($n^2$) time and space.

\subsection{Behavior Grid}
Animal residency is computed as a function of the depth of a particular cell (Restricted Vertical Habitat), and the cell's location (OU/RW modeling).  This computation takes a constant [O(1)] amount of time.  The space required to store a single residency value for each of $n^2$ cells is also O(1).  Therefore, the population of the Behavior Grid takes O(1) * $n^2$ = O($n^2$) time and space.


\subsection{Line of Sight Computation}
\label{bigOLoS}
As discussed in Section~\ref{LoSAlgorithm}, the LoS algorithm is given as:\newline
Step 1) Determine the $m$ intervening cells $C_{0...m}$ between $p$ and $q$\newline
Step 2) Compute the slopes between $p$ and $C_i$ for all $i$ in $[0 ... m]$\newline
Step 3) Choose the Critical Slope\newline
Step 4) Project a line from $p$ to $q$ along the Critical Slope to find $D_{max}$.\newline

\noindent\textbf{Step 1} finds intervening cells by ray tracing, which can be done in linear ($O(n)$) time.  Temporarily storing this list of intervening cells requires $O(n)$ space.  The number of cells considered ($n$), depends upon the distance between the receiver and the target cell\cite{rayTracing}.  Given that the Detection Area has a radius of $D_{range}$, and that receivers are located in the center of the Detection Area, the greatest distance between a receiver and a target cell in the Detection Area occurs between a receiver and the corner cells of the Detection Area (a distance of $\sqrt2*D_{range}$).  Thus, the number of cells considered when creating a list of intervening cells is $O(\sqrt2*D_{range}$ = $O(D_{range})$.\newline

\noindent\textbf{Step 2} computes the slopes for each of the $m$ cells in $C$.  Computing the slope between two points takes $O(1)$ time to compute and space to store.  Therefore computing $m$ slopes takes $O(1)*O(m)=O(m)$ time and space.  As stated above, $m$ is at worst $O(D_{range})$ on the Detection Area.  Thus, the slope computation is at worst $O(D_{range})$, requiring $O(D_{range})$ extra storage.\newline

\noindent\textbf{Step 3} chooses the largest slope amongst all $m$ slopes.  Therefore, $m$ items must be considered, each requiring $O(1)$ time to consider as the largest.  Since, $m$ is $O(D_{range})$ on the Detection Area, this step takes ($O(D_{range})$) time to compute, and $O(1)$ space to store.  \newline

\noindent\textbf{Step 4} is a direct computation, requiring $O(1)$ time to compute and $O(1)$ space to store the result.\newline

In total, the LoS computation time is $O(D_{range}) + O(D_{range}) + O(D_{range}) + O(1) = O(D_{range})$, requiring $O(D_{range}) + O(D_{range}) + O(1) + O(1) = O(D_{range})$ temporary storage space.
\begin{figure}[ht]
	\label{rayTracingImg}
	\centering
	\includegraphics[scale=0.3]{rayTracing.png}
	\caption{An illustration of how ray tracing is used within a 3D environment to identify potential visual obstructions.  Ray tracing is used to determine which cells potentially block the line of sight between the receiver-containing cell $p$, and the target cell $q$.  Shaded cells must be evaluated by the Line of Sight algorithm to determine which portion of the water column in $q$ that is visible from $p$. \cite{Akbarzadeh2013}}
\end{figure}

\subsection{Goodness Grid}
Recall that the population of the Goodness Grid is performed by Evaluation Algorithms (Section~\ref{evalAlgorithms}), which compute $ERT$ values for each of the $D_{range}^2$ cells in the Detection Area.  These ERT values are then summed (requiring $O(D_{range}^2)$  time) to provide the Goodness value for a single cell, out of the $n^2$ cells, in the Goodness Grid.  Thus, if the computational complexity of the ERT computation of Evaluation Algorithm $x$ is defined to be $E_x$, the computational complexity for populating the entire Goodness Grid is the time for ERT computation across the $D_{range}^2$ cells in the Detection Area plus the time for summing those $D_{range}^2$ ERT values, for each of the $n^2$ cells in the Goodness Grid.  Explicitly, this is $O(n^2 * [D_{range}^2* E_x + D_{range}^2]) = O(n^2 * D_{range}^2 * [E_x + 1]) = O(n^2 * D_{range}^2 * E_x)$. 


\subsubsection{Evaluation Algorithm 1}
\label{bigObias1}
Section~\ref{bias1} gives the $ERT$ computation for Evaluation Algorithm 1 as:\newline
$ERT_{1,i,j} = B_{i,j} * P_{Attenuation}(Range(i,j))$

Accessing the value $ B_{i,j}$, finding the distance from $i$ to $j$ ($Range(i,j))$), and computing the attenuation due to range ($P_{Attenuation}(Range(i,j))$) and finding the product of the two can each be done in $O(1)$ time.  Thus, the ERT computation for this algorithm is $O(1) + O(1) + O(1) = O(1)$.  This computation will require $O(1)$ additional pieces of temporary storage for each intermediary value and the resulting product.


\subsubsection{Evaluation Algorithm 2 \& 3}
\label{bigObias23}
As discussed in Section~\ref{bias3}, the ERT computation for Evaluation Algorithms 2 and 3 is given as:\newline
$ERT_{2/3,i,j} =  B_{i,j} * P_{observation} * P_{Attenuation}(Range(i,j))$\newline

As mentioned in Section~\ref{bigObias1}, the product of $B_{i,j}$ and $P_{Attenuation}(Range(i,j))$ can be computed in $O(1)$ time with $O(1)$ temporary storage.  The computation of $P_{observation}$ begins with determining the deepest depth ($D_{max}$) that can be seen, without bathymetric obstruction, in target cell ($i,j$).  As previously described in Section~\ref{bigOLoS}, the computation of $D_{max}$ can be done in $O(D_{range})$ time with $O(D_{range})$ temporary storage.  Finally, ($P_{observation}$) is computed as the definite integral over the shape function in $O(1)$ time.  Thus, the ERT computation for this algorithm is $O(1) + O(D_{range}) + O(1) = O(D_{range})$.  The algorithm will need to use $O(1)$ temporary storage for temporary storage of each of $B_{i,j}$, $P_{Attenuation}(Range(i,j))$, and $P_{observation}$.  An additional $O(D_{range})$ temporary storage space is required to compute Line of Sight, and $O(1)$ temporary space to store the resulting integral.  Thus, the total temporary storage required is given by $O(D_{range}+1+1+1+1)=O(D_{range})$.

%%\paragraph{Naive Method}
%%A naive solution to this question is to determine whether or not each cell in the three-dimensional grid can see the receiver.  Given that the volume of a sphere is $\frac{4/3}\pi r^{3}$, this solution would %%need to consider at least O($r^{3}$) cells, checking each for line of sight to the receiver.  As stated in Section~\ref{bigOLoS}, the time and space complexities of the LoS algorithm are both linear $O(r)$.  Thus, the time and space complexities for naively determining the line of sight to each cell in the Detection Area are $O(r^3) * O(r) = O(r^4)$.  

\subsection{Optimal Receiver Placement}
The selection of the top $R_{opt}$ receiver locations requires the Goodness Grid to have been populated.  As previously stated, the selection process is iterative (requiring that suppression (if selected) be applied after each selection), as suppression causes values in the Goodness Grid to be altered.  The process of selecting a single receiver location is $O(n^2)$ since the algorithm must consider approximately $n^2$ possible receiver locations at each iteration.  After each location selection, suppression must be applied to the chosen location.  Due to the variability of suppression algorithm complexity, a variable  ($E_{suppression}$) is used to represent the expected runtime of the suppression algorithm.  Thus the computation time required for each selection-suppression step is $O(n^2 + E_{suppression})$.  The number of iterations required to run is given by the number of optimal ($R_{opt}$) and projected ($R_{proj}$) receiver placements requested.  In total, the runtime of the Optimal Receiver Placement step is $[R_{opt} + R_{proj}] * [O(n^2 + E_{suppression})] = O([R_{opt} + R_{proj}]* [n^2 + E_{suppression}])$.


\subsection{Suppression}
As discussed in Section~\ref{suppression}, the suppression mechanic is applied after the placement of a receiver.  The suppression mechanic has several options, each affecting the time and space complexity of the mechanism.  To better capture the complexity of the algorithms involved, the following variables are defined: \newline

\noindent\begin{tabularx}{\linewidth}{@{}>{\hsize=.2\hsize}X>{\hsize=1.5\hsize}X@{}}
	
$D_{range}$& The Detection Range of a receiver.\\
$d_{detection}$& The edge size of the Detection Area, equal to $2*D_{range}+1$ cells.\\
$f_{supp}$ & The Suppression Range Factor.  A non-negative real number.\\
$r_{supp}$ & The Suppression Range. Equal to $f_{supp}*D_{range}$\\
$d_{supp}$ & The square root of the number of cells in the Suppression Area.  The edge dimension of the Suppression Area.  Equal to $2*r_{supp} + 1$ cells.\\
\end{tabularx}

\subsubsection{Static Suppression Algorithm}
As stated in Section~\ref{staticSuppression}, the Static Suppression Algorithm multiplies all cells within the Suppression Area by a scalar constant.  This is a scalar multiplication that can be done in place, requiring only $O(1)$ extra space, but a multiplication operation for each cell in the Suppression Area, which is $O(d_{supp} ^2) = O([2*f_{supp}*D_{range} + 1]^2) = O([4*f_{supp}^2*D_{range}^2 + 4*f_{supp}*D_{range} + 1]) = O(f_{supp}^2*D_{range}^2)$.  In most cases (excluding those where very sparse acoustic networks are desired), the Suppression Range Factor, $f_{supp}$, will be rather small (less than 2.0), which reduces the runtime complexity to $O(D_{range}^2)$.

\subsubsection{Exact Suppression}
As stated in Section~\ref{exactSuppression}, the Exact Suppression Algorithm uses the user-specified Evaluation Algorithm to calculate the number of transmissions that have already been observed and should be discounted from future consideration.  Here, $E_{x}$ denotes the expected runtime for Evaluation Algorithm x's computation of a single cell's ERT and, $T_{x}$ denotes the expected temporary storage requirement for Evaluation Algorithm $x$'s single-cell ERT computation.  The Exact Suppression Algorithm is broken into three distinct steps: Suppression Area ERT computation, Behavior Grid Update, and Goodness Grid Recalculation.
\newline
\newline 
\textbf{Suppression Area ERT Computation}\newline
The Suppression Algorithm first determines the ERT for each cell within the Suppression Area, and stores them in the ERT Area, a temporary Grid with the same dimension as the Suppression Area.  This step is similar to the Evaluation process, with the exception that the Suppression Area's dimensions differ from those of the Detection Area by a factor of $f_{supp}$.  The computation of the ERT Area requires $E_{x}$ time and  $T_x$ temporary space for each of the $d_{supp}^2$ cells in the Suppression Area.  An additional  $O(1)$ temporary space will be needed to store the resulting ERT values for each cell.  Therefore, the ERT computation will require $O(E_{x} * d_{supp}^2)$ time and $O((1 + T_x)* d_{supp}^2) = O(d_{supp}^2 * T_x)$ temporary storage.
\newline
\newline
\textbf{Behavior Grid Update}\newline
Next, the suppression algorithm subtracts the ERT Area from the corresponding area in the Behavior Grid.  This is a simple subtraction operation between each of the $d_{supp}^2$ pairs of corresponding cells in the Grids.  Because each subtraction takes $O(1)$ time and temporary storage, the time and space complexities are both given by $O(1) * O(d_{supp}^2) = O(d_{supp}^2)$.

\textbf{Goodness Grid Recalculation}\newline
Finally, the Goodness values of all cells within Detection Range of the Suppression Area must have their Goodness value updated (as the ERT of one or more cells within their Detection Area has just been reduced).  The area affected by suppression, and therefore requiring recalculation, is a square with edge diameter $2 * (d_{supp} + D_{range}) + 1$.  Each cell in this area will require  $O(d_{detection}^2 * E_x)$ time to re-compute its Goodness Value.  The total time for updating the Goodness Grid is then: $O([2 * (d_{supp} + D_{range}) + 1]^2 * O(d_{detection}^2 * E_x)) = O((d_{supp} + D_{range})^2 * d_{detection}^2 * E_x)$.  The Goodness Grid update will require $d_{detection}^2 * (T_x + 1) = O(d_{detection}^2 * T_x)$ temporary storage to compute and store intermediate ERT values.  This temporary storage can be recycled for each sequential Goodness Cell computation.


The total computation time for the Exact Suppression Algorithm is then given by O(ERT Computation) + O(Behavior Grid Update) + O(Goodness Grid Recalculation). This is a total of $O(E_{x} * d_{supp}^2) + O(d_{supp}^2) + O((d_{supp} + D_{range})^2 + d_{detection}^2 * E_x)$ time.  The algorithm will also require $O(MAX(d_{supp}^2 , d_{detection}^2 * T_x)$ temporary storage, which can be recycled between each of the above stages.
