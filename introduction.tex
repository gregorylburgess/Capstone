\chapter{Introduction}

\section{Observation Networks}

\section{State of the Art}
\subsection{The Cost of Data}

VR2's cost ~\$1.5k each, tags cost ~\$350 each.  http://www.gulfcounty-fl.gov/pdf/882532513025603.pdf 


\subsection{"Rules of Thumb for Sensor Placement"}
Heuple 
\subsection{Existing Metrics}
\subsection{Scale of Experiments}

\section{Requirements}
\subsection{Scope of Tool}
\subsection{Supported Workflows}
\subsection{Bathymetric File Support}
\subsection{Bathymetric Shadowing}
\subsection{Modeling Animal Movement and Habitat}
\paragraph{Ornstein-Uhlenbeck}
This is a paragraph about OU.
\paragraph{Random Walk}
\subsection{Evaluation of Sensor Emplacements}
\subsection{Selection of Optimal Emplacements}


Here is a picture in figure \ref{fig:example-1}.

\begin{figure}[htbp]
  \centering
  \caption{An example of included Encapsulated PostScript (EPS).}
  \label{fig:example-1}
\end{figure}

Using the package we get the much nicer \url{<http://www.hotwired.com/
webmonkey/98/16/index2a.html>} which LaTeX can handle just fine. Even better,
the parameter to {\tt $\backslash$url} can have spaces inserted anywhere so you
can make the LaTeX source lines in your text editor wrap nicely.

A few notes. It is recommended that you enclose your URLs in ``$<>$'' to ensure
that any punctuation around the URL won't be confused as part of the URL. You
can use URLs in your bibliography too (see the {\tt uhtest.bib} file for an
example). Finally, if you need to use a tilde in your URL then things are a
little trickier. One way to do it is like this:
\url{<http://www.dartmouth.edu/}$\sim$\url{jonh/ff-cache/1.html>}. The {\tt
$\backslash$url} style uses math mode internally, so we break the URL into two
pieces, and stick a tilde from math mode inbetween the two parts.

